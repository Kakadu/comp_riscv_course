% !TEX TS-program = xelatex
% !TeX spellcheck = ru_RU
% !TEX root = sel2dag1.tex

\documentclass[tikz,border=3.14mm]{standalone}

\xdefinecolor{mycolor}{RGB}{62,96,111} % Neutral Blue
\colorlet{bancolor}{mycolor}

\usepackage{tikz}
\usetikzlibrary{arrows,backgrounds,calc,trees}

\pgfdeclarelayer{background}
\pgfsetlayers{background,main}


\newcommand{\convexpath}[2]{
    [
    create hullnodes/.code={
        \global\edef\namelist{#1}
        \foreach [count=\counter] \nodename in \namelist {
            \global\edef\numberofnodes{\counter}
            \node at (\nodename) [draw=none,name=hullnode\counter] {};
        }
        \node at (hullnode\numberofnodes) [name=hullnode0,draw=none] {};
        \pgfmathtruncatemacro\lastnumber{\numberofnodes+1}
        \node at (hullnode1) [name=hullnode\lastnumber,draw=none] {};
    },
    create hullnodes
    ]
    ($(hullnode1)!#2!-90:(hullnode0)$)
    \foreach [
    evaluate=\currentnode as \previousnode using \currentnode-1,
    evaluate=\currentnode as \nextnode using \currentnode+1
    ] \currentnode in {1,...,\numberofnodes} {
        let
        \p1 = ($(hullnode\currentnode)!#2!-90:(hullnode\previousnode)$),
        \p2 = ($(hullnode\currentnode)!#2!90:(hullnode\nextnode)$),
        \p3 = ($(\p1) - (hullnode\currentnode)$),
        \n1 = {atan2(\y3,\x3)},
        \p4 = ($(\p2) - (hullnode\currentnode)$),
        \n2 = {atan2(\y4,\x4)},
        \n{delta} = {-Mod(\n1-\n2,360)}
        in
        {-- (\p1) arc[start angle=\n1, delta angle=\n{delta}, radius=#2] -- (\p2)}
    }
    -- cycle
}

\begin{document}
\thispagestyle{empty}



\begin{tikzpicture}[every tree node/.style={draw,circle}
    , sibling distance=25pt
    , level distance=30pt
    , leaf/.style={circle,fill=blue,fill opacity=0.3,text opacity=1}
    , dot/.default = 4pt % size of the circle diameter
    %, ->/.style={thick,->}
    ]
    \node[leaf] at (-1,-1) (M7) {$\times$};
    \node[leaf] at ( 1,-1) (M9) {$ld$};
    \node[leaf,minimum size=9pt] at ( 0,-2) (M6) {$+$};
    \node[leaf] at (-1,-3) (M1) {$a$};
    \node[leaf] at ( 1,-3) (M2) {$b$};
    \node[leaf] at (-2,-2) (M3) {$c$};
    \begin{pgfonlayer}{background}
        \fill[red,opacity=0.3] \convexpath{M6,M7}{8pt};
        \fill[red,opacity=0.7] \convexpath{M6,M9}{8pt};
    \end{pgfonlayer}
    \node at ( 1.2,-2) (labM10) {$m_{10}$};
    \draw[red,opacity=0.7] (labM10) --  (.8,-1.5);
    \node at ( -.2,-1) (labM8) {$m_8$};
    \draw[red,opacity=0.7] (labM8) --  (-.28,-1.1);
    \node at (1.8,-1) (labM7) {$m_9$};
    \node at (-1.8,-1) (labM7) {$m_7$};
    \node at (-2,-2.5) (labM3) {$m_3$};
    \node at (-.3,-3) (labM1) {$m_1$};
    \node at (1.8,-3) (labM2) {$m_2$};

\draw[->,thick] (M1) --  (M6);
\draw[->,thick] (M2) --  (M6);
\draw[->,thick] (M3) --  (M7);
\draw[->,thick] (M6) --  (M7);
\draw[->,thick] (M6) --  (M9);

\end{tikzpicture}



\begin{tikzpicture}[every tree node/.style={draw,circle}
    , sibling distance=25pt
    , level distance=30pt
    , leaf/.style={circle,fill=blue,fill opacity=0.3,text opacity=1}
    , dot/.default = 6pt % size of the circle diameter
    ]
\node[leaf] at ( 0, 0) (M7) {$\times$};
\node[leaf] at ( 1,-1) (M4) {$t$};
\node[leaf] at (-1,-1) (M3) {$c$};
\node[leaf] at (-2,-1) (M2) {$b$};
\node[leaf] at (-3, 0) (M6) {$+$};
\node[leaf] at (-4,-1) (M1) {$a$};
\node[leaf] at ( 2, 0) (M9) {$ld$};
\node[leaf] at ( 2,-1) (M5) {$t$};
\begin{pgfonlayer}{background}
%   \fill[red,opacity=0.3] \convexpath{a,g,b}{8pt};
%    \fill[red,opacity=0.3] \convexpath{m6,m7}{8pt};
%    \fill[blue,opacity=0.3] \convexpath{m6,m7,m9}{10pt};
%    \fill[blue,opacity=0.3] \convexpath{m3,m3}{10pt};
\end{pgfonlayer}
\draw[->,thick] (M1) --  (M6);
\draw[->,thick] (M2) --  (M6);
\draw[->,thick] (M3) --  (M7);
\draw[->,thick] (M4) --  (M7);
\draw[->,thick] (M5) --  (M9);
%\draw[blue] (labM8) --  (.20,-1.3);
\node at (-4,-2) (labM1) {$m_1$};
\node at (-2,-2) (labM2) {$m_2$};
\node at (-1,-2) (labM3) {$m_3$};
\node at ( 1,-2) (labM4) {$m_4$};
\node at ( 2,-2) (labM5) {$m_5$};
\node at (-4, 0) (labM6) {$m_6$};
\node at (-1, 0) (labM7) {$m_7$};
\node at ( 3, 0) (labM9) {$m_9$};
\end{tikzpicture}

\end{document}